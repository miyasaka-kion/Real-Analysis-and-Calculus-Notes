\documentclass[11pt]{article}
\usepackage{amsmath}
\usepackage{geometry}
\usepackage{amsfonts}
\usepackage{tcolorbox}
\def\it#1{\textit{#1}}
\def\bm#1{\textbf{#1}}
\newcommand\df{\mathrel{\overset{\makebox[0pt]{\mbox{\normalfont\tiny\sffamily def}}}{=}}}
\newcommand*\diff{\mathop{}\!\mathrm{d}}
\newcommand*\Diff[1]{\mathop{}\!\mathrm{d^#1}}
\newcommand*\eps{\varepsilon}

\geometry{a4paper,scale=0.75}
\title{\huge \bfseries Real  Analysis and Calculus Note}
\author{Kion \textsc{Miyasaka}}
\begin{document}
\maketitle

\section{Sequence}

\begin{tcolorbox}[colback=blue!5,colframe=black!5!red,title=\bm{Axiom of Completeness}]
	A nonempty set of real numbers that is bounded above has a least upper bound, i.e. supremums of bounded sets are real numbers.
\end{tcolorbox}


\begin{tcolorbox}[colback=blue!5,colframe=black!75!green,title=\bm{Monotone Subsequence}]
	Every sequence contains a monotonic subsequence.
	\begin{tcolorbox}[colback=blue!5,colframe=black!75!black,title=\bm{Monotone Convergence Theorem}]
	Suppose that $\left\{x_{n}\right\}$ is a monotonic sequence. Then, $\left\{x_{n}\right\}$ is convergent if and only if $\left\{x_{n}\right\}$ is bounded.
\end{tcolorbox}

\end{tcolorbox}


\begin{tcolorbox}[colback=blue!5,colframe=black!75!black,title=\bm{Bolzano-Weierstrass}]
Every bounded sequence contains a convergent subsequence.
\end{tcolorbox}
 
 \begin{tcolorbox}[colback=blue!5,colframe=black!75!black,title=\bm{Cauchy Convergence Criterion}]
 A sequence $\left\{x_{n}\right\}$ is convergent iff for each $\varepsilon>0$ there exists an integer $N$ with the property that
$$
\left|x_{n}-x_{m}\right| \leq \varepsilon
$$
for all $n \geq N$ and $m \geq N$.	
 \end{tcolorbox}
 
 \qquad Completeness Axiom of reals
 
$\Longrightarrow$ Monotonic Convergence Theorem

$\Longrightarrow$ Bolzano-Weierstrass Theorem

$\Longrightarrow$ Cauchy Convergence Criterion


\newpage
\section{Series}
\subsection{Conditions of convergence}

If $\sum_{k=1}^{\infty} a_{k}$ converges, then $a_{k} \rightarrow 0$ as $k \rightarrow \infty$.

If the series $\sum_{k=1}^{\infty}\left|a_{k}\right|$ converges, then so does the series $\sum_{k=1}^{\infty}a_{k}$


A series $\sum_{k=1}^{\infty} a_{k}$ is said to be absolutely convergent if $\sum_{k=1}^{\infty}\left|a_{k}\right|$ converges.


In order to distinguish convergence from absolute convergence, we refer to the former as non-absolutely convergence, or conditional convergence.


A series $\sum_{k=1}^{\infty} a_{k}$ is said to be non-absolutely (or conditional) convergent if it converges but the series $\sum_{k=1}^{\infty}\left|a_{k}\right|$ diverges.

\subsection{Properties of Convergent Series}
\subsubsection{Dirichlet's Theorem: rearrangements of series}

Let $\sum_{n=1}^{\infty} a_{n}$ be an absolutely convergent series of real numbers. If $\left\{b_{n}\right\}$ is any rearrangement of $\left\{a_{n}\right\}$ then

1. $\sum_{n=1}^{\infty} b_{n}$ is an absolutely convergent series.

2. $\sum_{n=1}^{\infty} b_{n}=\sum_{n=1}^{\infty} a_{n}$

\subsubsection{Conditional convergence}
If the series $\sum_{n=1}^{\infty} a_{n}$ converges but does $n o t$ converge absolutely (i.e. $\sum_{n=1}^{\infty} a_{n}$ is conditionally convergent) and $\gamma \in \mathbb{R}$ is any real number, then there exists a rearrangement $\left\{b_{n}\right\}$ of the sequence $\left\{a_{n}\right\}$ so that
$$
\sum_{n=1}^{\infty} b_{n}=\gamma
$$

\subsection{Converge Tests}
\subsubsection{Null test}
If the terms of the series $\sum_{k=1}^{\infty} a_{k}$ do not converge to zero, then the series diverges.

\subsubsection{Comparison test}
Given two series $\sum_{k=1}^{\infty} a_{k}$ and $\sum_{k=1}^{\infty} b_{k}$ such that $0 \leq a_{k} \leq b_{k}$ for all $k$.

1. If the larger series converges, then so does the smaller series.

2. If the smaller series diverges, then so does the larger series.

\subsubsection{Ratio test}
If terms of the series $\sum_{k=1}^{\infty} a_{k}$ are all positive and the ratios
$$
\lim _{k \rightarrow \infty} \frac{a_{k+1}}{a_{k}}<1
$$
then the series is convergent.
\subsubsection{Root test}
If terms of the series $\sum_{k=1}^{\infty} a_{k}$ are all nonnegative and the roots
$$
\lim _{k \rightarrow \infty} \sqrt[k]{a_{k}}<1
$$
then the series is convergent.

\subsubsection{Integral test}

Let $f$ be a nonnegative decreasing function on $[1, \infty) .$ Then
$$
\lim _{X \rightarrow \infty} \int_{1}^{X} f(x) d x
$$
converges if and only if the series $\sum_{k=1}^{\infty} f(k)$ converges.

\paragraph{Proof} since $f$ is decreasing we have
$$
\int_{k}^{k+1} f(x) d x \leq f(k) \leq \int_{k-1}^{k} f(x) d x
$$
Thus
$$
\int_{1}^{n+1} f(x) d x \leq \sum_{k=1}^{n} f(k) \leq f(1)+\int_{1}^{n} f(x) d x
$$
The series converges if and only if the partial sums are bounded.

\subsubsection{Alternating Series test}
The series
$$
\sum_{k=1}^{\infty}(-1)^{k-1} a_{k}
$$
where the terms alternate in sign, converges if the sequence $\left\{a_{k}\right\}$ decreases monotonically to zero.

\newpage
\section{Power Series}
A power series 
$$
f(x) \df \sum _{n=0}^\infty a_n x^n, x\in S
$$
where $S$ will make sense.

\subsection{Radius of Convergence}
Given a power series $\sum_{n=0}^{\infty} a_{n} x^{n},$ either it converges absolutely for all $x \in \mathbb{R},$ or there exists $R \in[0, \infty)$ such that

(1) it converges absolutely when $|x|<R$

(2) it diverges when $|x|>R$.

\paragraph{Remark}
We can restate the theorem as
$$
(-R, R) \subseteq S \subseteq[-R, R]
$$
and the power series converges absolutely in $(-R, R) .$ In particular we see that $S$ is always an interval.

\subsection{Convergence Test (of Power Series)}
\subsubsection{Ratio test (of Power Series)}
Consider the power series
$$
\sum_{n=0}^{\infty} a_{n} x^{n}
$$
Suppose that
$$
\frac{\left|a_{n+1}\right|}{\left|a_{n}\right|} \rightarrow \ell, \quad \text { as } \quad n \rightarrow \infty
$$
Then
$$
R=\left\{\begin{array}{ll}
0 & \text { if } \quad \ell=\infty \\
\frac{1}{\ell} & \text { if } \quad \ell \in \mathbb{R} \backslash\{0\} \\
\infty & \text { if } \quad \ell=0
\end{array}\right.
$$
\subsubsection{Root test (of Power Series)}

Consider the power series
$$
\sum_{n=0}^{\infty} a_{n} x^{n}
$$
Suppose that
$$
\left|a_{n}\right|^{\frac{1}{n}} \rightarrow \ell, \quad \text { as } \quad n \rightarrow \infty
$$
Then
$$
R=\left\{\begin{array}{ll}
0 & \text { if } \quad \ell=\infty \\
\frac{1}{\ell} & \text { if } \quad \ell \in \mathbb{R} \backslash\{0\} \\
\infty & \text { if } \quad \ell=0
\end{array}\right.
$$

\newpage
\section{Maclaurin and Taylor Series}
\paragraph{Definition}
If the function $f$ has a power series representation on the interval $(c-R, c+R),$ then the power series
$$
\begin{aligned}
f(x) &=\sum_{n=0}^{\infty} \frac{f^{(n)}(c)}{n !}(x-c)^{n} \\
&=\frac{f(c)}{0 !}+\frac{f^{\prime}(c)(x-c)}{1 !}+\frac{f^{\prime \prime}(c)(x-c)^{2}}{2 !}+\frac{f^{\prime \prime \prime}(c)(x-c)^{3}}{3 !}+\cdots
\end{aligned}
$$
is called the \bm{Taylor Series of the function $f$ about $c$}. In the particular case that $c=0,$ then Taylor series of $f$ is usually called the Maclaurin series of $f:$
$$
\begin{aligned}
f(x) &=\sum_{n=0}^{\infty} \frac{f^{(n)}(0)}{n !} x^{n} \\
&=\frac{f(0)}{0 !}+\frac{f^{\prime}(0) x}{1 !}+\frac{f^{\prime \prime}(0) x^{2}}{2 !}+\frac{f^{\prime \prime \prime}(0) x^{3}}{3 !}+\cdots
\end{aligned}
$$
\\

\subsection{Things must be Memorized}
1. For any $x \in \mathbb{R}$
$$
e^{x}=1+x+\frac{x^{2}}{2}+\cdots+\frac{x^{n}}{n !}+\cdots=\sum_{n=0}^{\infty} x^{n} / n !
$$
2. For any $x \in \mathbb{R}$
$$
\sin x=x-\frac{x^{3}}{3 !}+\frac{x^{5}}{5 !}+\cdots+(-1)^{n} \frac{x^{2 n+1}}{(2 n+1) !}+\cdots=\sum_{n=0}^{\infty}(-1)^{n} x^{2 n+1} /(2 n+1) !
$$
3. For any $x \in \mathbb{R}$
$$
\cos x=1-\frac{x^{2}}{2}+\frac{x^{4}}{4 !}+\cdots+(-1)^{n} \frac{x^{2 n}}{(2 n) !}+\cdots=\sum_{n=0}^{\infty}(-1)^{n} x^{2 n} /(2 n) !
$$
4. The Binomial Theorem: for any $\alpha \in \mathbb{R}$ and $x$ such that $|x|<1$
$$
\begin{aligned}
(1+x)^{\alpha} &=1+\alpha x+\frac{\alpha(\alpha-1)}{2} x^{2}+\cdots+\frac{\alpha(\alpha-1) \ldots(\alpha-n+1)}{n !} x^{n}+\cdots \\
&=\sum_{n \geq 0} \frac{\alpha(\alpha-1) \cdots(\alpha-n+1)}{n !} x^{n}
\end{aligned}
$$
5. From $4 .$ we have, for any $x$ such that $|x|<1$,

\begin{equation}
\begin{aligned}
\frac{1}{1-x}&=1+x+x^{2}+x^{3}+\cdots+x^{n}+\cdots=\sum_{n=0}^{\infty} x^{n} \\
\frac{1}{1+x}&=1-x+x^{2}-x^{3}+\cdots+(-1)^{n} x^{n}+\cdots=\sum_{n=0}^{\infty}(-x)^{n}
\end{aligned}
\end{equation}

6. For any $x$ such that $|x|<1$
$$
\log (1+x)=x-\frac{x^{2}}{2}+\frac{x^{3}}{3}-\frac{x^{4}}{4}+\cdots+(-1)^{n+1} \frac{x^{n}}{n}+\cdots=\sum_{n \geq 1}(-1)^{n+1} \frac{x^{n}}{n}
$$

\newpage
\section{Things Must be Mentioned about Series}
\subsection{$\sum a_n $ converges$ \iff \sum a_n^3$ converges}
\subsubsection{$\not \Rightarrow$} 
$\sum a_n$ converges \bm{does not} imply $\sum a_n^3$ converges in general for non positive $a_n$. For an $m$, write as $m=3n+k$ where $0\leq k<3$ and define $a_{3n+k}= \frac{b_k}{(n+1)^{1/3}}$ where $b_0=2$ and $b_1,b_2=-1$. Then $a^3_n$ in general looks like
$$
\frac{8}{1}, \frac{-1}{1}, \frac{-1}{1}, \frac{8}{2}, \frac{-1}{2}, \frac{-1}{2}, \frac{8}{3}, \frac{-1}{3}, \frac{-1}{3}, \ldots
$$
which has partial sums $S_{3n}=6\sum _{i=1}^{\infty} 1/i$ diverging.

\subsubsection{$\not \Leftarrow$}
$$
\sum \frac{1}{k^3}\quad \textit{converges} \not \Rightarrow \sum \frac{1}{k}\quad \textit{converges} 
$$

%\subsubsection{Further More}
%Let $C$ be an arbitrary subset of the positive integers ($C$ may be finite or infinite). Then there is a sequence $a_1,a_2,a_3,...$ of real numbers (of course, depending on $C$) such that for any positive integer $l$,
%$$
%\sum a_n^{2l-1}
%$$
%converges if and only if $l\in C$.

\newpage
\section{Basic Integration Methods}
\subsection{Some Other Things}
\subsubsection{Hyperbolic Function}
\paragraph{Definition}
\begin{equation}
\sinh x \df \frac{e^x-e^{-x}}{2}\qquad
\cosh x \df \frac{e^x+e^{-x}}{2}
\end{equation}
\subsubsection{Identities of Trigonometric Functions}
\paragraph{Pythagorean Identities}
\begin{equation}
\begin{aligned}
\sin^2 \theta + \cos^2 \theta=& 1\\
\tan^2 + 1=&\sec^2 \theta \\
1+ \cot ^2=& \csc^2 \theta 	
\end{aligned}
\end{equation}
\subsubsection{Recite these Equations}
For integrate is the inverse of the derivative, we just have to recite some basic rules about the derivative.
Equation below request to be recited.
\begin{equation}
\begin{aligned}
(\arcsin x)'&=\frac{1}{\sqrt{1-x^2}}\\
(\arccos x)'&=-\frac{1}{\sqrt{1-x^2}}\\
(\arctan x)'&=\frac{1}{1+x^2}\\
(\tan x)'&=\sec^2 x\\
(\cot x)'&=-\csc^2 x\\
(\sinh x)'&= \cosh x\\ 
(\cosh x)'&= \sinh x\\
\end{aligned}
\end{equation}

\subsection{Some Classic Integrations}
The following equation have some interesting conclusion.
\begin{equation}
\begin{aligned}
\int \sec x\diff x &=\ln  \left|\frac{\cos \frac{x}{2} + \sin \frac{x}{2}}{\cos \frac{x}{2} -\sin \frac{x}{2}}\right|+C\\
&= \ln \left| \frac{1+\sin x}{\cos x}\right|+C \\
&= \ln \left| \sec x+ \tan x \right| + C
\end{aligned}
\end{equation}

\section{Riemannn Integrable}
Continuous\footnote{On closed set} $\Rightarrow$ Riemann Integrable $\Rightarrow$ Boundness

\quad \quad \qquad \qquad \qquad \quad $\Uparrow$

\quad \qquad \qquad \qquad Monotone 


\section{Some Particular Functions}
\subsection{Weierstrass function}
the Weierstrass function is an example of a real-valued function that is continuous everywhere but differentiable nowhere.
In Weierstrass's original paper, the function was defined as a Fourier series:
$$
f(x)=\sum_{n=0}^{\infty} a^{n} \cos \left(b^{n} \pi x\right)
$$
where $0<a<1, b$ is a positive odd integer, and
$$
a b>1+\frac{3}{2} \pi
$$

\subsection{Volterra's function}
The function is defined by making use of the Smith Volterra Cantor set and "copies" of the function defined by $f(x)=x^{2}\sin(1/x)$ for $x\neq 0$ and $f(0)=0$.

\begin{itemize}
	\item $V$ is differentiable everywhere
	\item The derivative $V'$ is bounded everywhere
	\item The derivative is \bm{not Riemann-integrable}.
\end{itemize}


\end{document}
